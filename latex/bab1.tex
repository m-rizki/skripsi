%!TEX root = ./template-skripsi.tex
%-------------------------------------------------------------------------------
% 								BAB I
% 							LATAR BELAKANG
%-------------------------------------------------------------------------------

\chapter{PENDAHULUAN}
\section{Latar Belakang Masalah}

Salah satu organ yang sangat penting pada tubuh manusia adalah kulit. Ketika kulit terluka, diperlukan perawatan luka yang baik agar tidak terjadi infeksi. Luka adalah keadaan di mana fungsi anatomis kulit normal mengalami kerusakan akibat proses patologis yang berasal dari internal maupun eksternal dan mengenai organ tertentu. Proses penyembuhan luka terjadi melalui beberapa fase, yaitu: hemostasis (beberapa jam pasca-terjadinya luka), inflamasi (1 – 3 hari), proliferasi (4 – 21 hari), dan \emph{remodelling} (21 hari – 1 tahun). Fase-fase penyembuhan luka terjadi secara bertahap, namun dapat terjadi secara bersamaan (\emph{overlap}) \citep{pat2018:1}. %fine

Luka kronis adalah kondisi di mana luka mengalami proses penyembuhan yang tidak normal dengan durasi fase-fase yang sesuai \citep{landen2016transition:2}, kondisi ini dapat berkaitan dengan berbagai faktor yang memperlambat penyembuhan luka seperti adanya penyakit kronis, insufisiensi vaskuler, diabetes, gangguan nutrisi, penuaan, dan berbagai faktor lokal pada luka (tekanan, infeksi, dan edema). Secara umum, luka kronis dapat terjadi akibat ulkus vena, ulkus arteri, ulkus dekubitus, dan ulkus diabetik \citep{zhao2016inflammation:3}. %fine

Setiap tahunnya prevalensi luka secara umum mengalami peningkatan. 1.4 juta orang dewasa dirawat karena luka kekerasan di tahun 2000 sampai 2010, 1,6\% di antaranya dirawat di Unit Gawat Darurat (UGD) di Amerika Serikat \citep{monuteaux2017cross:4}. Berdasarkan hasil Riset kesehatan dasar (Riskesdas) tahun 2018, prevalensi luka di Indonesia selalu mengalami peningkatan dari tahun 2007 sebanyak 7,5\%, tahun 2013 sebanyak 8,2\%, dan di tahun 2018 sebanyak 9.2\% \citep{hasilkesehatan2018riset}. %fine

Luka kronis bisa jadi merupakan komplikasi pada penderita Diabetes Melitus (DM). Sebanyak 15\% dari seluruh populasi penderita DM memiliki komplikasi berupa luka diabetes \citep{fard2007assessment:6}. Prevalensi Diabetes Melitus di Indonesia kategori penduduk umur 15 tahun keatas pada tahun 2018 adalah 2\% \citep{hasilkesehatan2018riset}. Pada tahun 2030 diprediksi meningkat menjadi 21,3 juta orang. Indonesia menempati peringkat keempat jumlah penderita DM terbanyak di dunia \citep{wild2004global:7}. %fine

Berdasarkan hasil prevalensi, luka kronis menjadi permasalahan bagi perawat luka dan instansi kesehatan terkait. Selain itu diperlukan penanganan khusus dalam proses pemulihan luka kronis. Permasalahan luka kronis menghadirkan kesulitan berat bagi yang menderita kondisi ini dan beban keuangan untuk industri kesehatan. Di sisi klien dapat berdampak pada penurunan kualitas hidup, ketidakmampuan untuk melakukan fungsi tubuh secara optimal, serta tingginya kebutuhan finansial. Bahkan dalam beberapa kasus dapat menyebabkan amputasi dan kematian. Bagi instansi kesehatan terkait akan memberikan dampak pada tingginya pembayaran asuransi kesehatan dikarenakan frekuensi perawatan luka yang dilakukan paling tidak 2,4 kali per minggu di mana menghabiskan 66\% waktu perawat luka \citep{HSE2007:8}. Di Amerika Serikat , luka kronis setiap tahunnya menelan biaya \$20 miliar dan memengaruhi 5,7 juta orang \citep{brown2018wearable:9}. Berdasarkan data dari alodokter.com, biaya perawatan luka di Indonesia berkisar mulai dari antara Rp61.500,00–Rp267.000,00 belum termasuk biaya balutan. %fine

Salah satu tanggung jawab perawat luka profesional adalah melakukan pengkajian pada luka, di mana hasil pengkajian tersebut bermanfaat untuk menentukan pemberian balutan luka yang tepat, memonitor perbaikan luka dan mencegah komplikasi sehingga perawatan akan menjadi \emph{cost effective} \citep{benbow2016best:10}. %fine

% jelaskan pentingnya deteksi tepi pada pengkajian luka (bab 1).
Salah satu hal yang mendasar dalam proses penyembuhan luka kronis adalah melihat ukuran luka, dan umumnya hal ini adalah kriteria pertama yang harus dipertimbangkan dalam proses \emph{assessment} luka yang mana hal ini memiliki peran penting di antaranya memantau laju dan kemajuan penyembuhan, mengevaluasi efektivitas perawatan, dan mengidentifikasi luka. Selain itu perubahan ukuran luka memungkinkan kita dalam mengamati tingkat penutupan, waktu penutupan, pelebaran, dan wawasan lain yang merupakan indikator untuk memprediksi status penyembuhan \citep{carrion2022automatic}. Masih digunakannya metode konvensional seperti mengukur menggunakan penggaris memiliki tingkat akurasi dan reliabilitas rendah sehingga perawat terkesan lambat dalam memberikan perawatan dibandingkan dengan profesi kesehatan yang lain seperti dokter. Sebuah studi menyatakan standar teknik pengukuran perawatan yang melibatkan penggunaan perkiraan penggaris dan mata telanjang, memiliki tingkat kesalahan 44\% \citep{budman2015design:11}. Untuk mengatasi ketidakakuratan pengukuran manual, maka metode pengukuran keliling luka berbasis analisa citra (\emph{image}), khususnya citra biomedis (\emph{biomedical image}) dan citra medis (\emph{medical image}) perlu dikembangkan. %fine

Saat ini penelitian \emph{state of the art} telah dilakukan untuk analisis luka. Banyak studi yang menjelaskan evaluasi luka menggunakan citra luka untuk mengetahui status luka. Citra medis dapat mengirimkan informasi lebih banyak untuk para ahli kesehatan daripada deskripsi subjektif yang cenderung menimbulkan kesalahan interpretasi. Lebih jauh lagi, gambar citra luka dapat digunakan untuk mentransmisi informasi tentang status penyembuhan untuk konsultasi medis di lokasi pedalaman. Pada sebuah percobaan tahun 2013, ditemukan nilai tinggi yang dihubungkan ke galeri foto luka di aplikasi mobile dan pelacakan luka melalui progresi grafis. Sehingga, untuk meningkatkan fitur citra diperlukan pengembangan algoritma analisis citra untuk penentuan ukuran dan warna dari foto luka yang diambil dari kamera telepon pintar atau kamera tablet \citep{poon2015algorithms:12}. %fine

Ketika sebuah foto diambil dengan pose yang sama oleh kamera yang berbeda, warna yang tersimpan mungkin berbeda, ini merupakan salah satu penelitian paling awal untuk analisis luka. Salah satu solusi untuk masalah tersebut adalah menggunakan format \emph{device independent} sRGB. Semua vendor kamera setidaknya menawarkan mode ini tetapi default ke format RGB mereka sendiri. Menggunakan \emph{chart} referensi warna ketika gambar citra luka diambil, Poucke et. al., berhasil melakukan mekanisme kalibrasi antara perangkat \emph{device dependent} RGB ke sRGB dengan mentransformasi citra terlebih dahulu menjadi ruang warna CIE \emph{colorimetric}, kemudian ke sRGB melalui serangkaian masalah optimisasi \citep{van2010automatic:13}. %fine

Upaya untuk menentukan batas luka oleh kamera telah dilaporkan dalam beberapa penelitian. Wang et. al. mengembangkan kotak pengambilan gambar dengan kamera \emph{smartphone} \& dua cermin, untuk menangkap gambar ulkus kaki dasar. Batas selanjutnya disempurnakan dengan algoritma \emph{mean-shift} kemudian disetel lagi dengan \emph{Region Adjacency Graph}. Setelah batas diperoleh, K-means dijalankan untuk mengukur rasio warna RYB. Metode ini dievaluasi pada 34 pasien yang berbeda di klinik Worcester. Kemudian penelitian Wang, masih berkonsentrasi pada penentuan batas luka tetapi dengan SVM \citep{wang2016area:15}. Pelabelan manual kumpulan data 100 luka yang menghasilkan 10.000 wilayah dilakukan oleh tim dokter (tiga ekspert) di sekolah kedokteran UMASS. Metode kerjanya sebagai berikut, pertama citra disegmentasi menjadi superpiksel dengan SLIC. Kemudian deskriptor warna \& tekstur diekstraksi untuk persiapan setiap tahap SVM. Pada tahap pertama, warna \& Tas kata diekstraksi dengan DSIFT. Selama tahap kedua, warna \& tekstur \emph{wavelet} diekstraksi. Tahap SVM pertama menjalankan k-binary SVM \emph{classifier} dilatih pada set citra yang berbeda. Pada tahap kedua, set kesalahan klasifikasi dilatih lagi dengan SVM biner. Setelah selesai, hasilnya sekali lagi disempurnakan dengan \emph{Conditional Random Field}. Meskipun memberikan hasil yang memuaskan, semua percobaan diuji pada ulkus kaki (\emph{close wound}) \citep{wang2014smartphone:14}. %fine

Friesen dari University of Manitoba memimpin tim peneliti untuk melakukan serangkaian penelitian dalam analisis luka. White et. al., sebagai tim peneliti pertama yang berkonsentrasi untuk mengukur ukuran luka ulkus tekan dilakukan dalam tiga langkah. Pertama, mengukur jarak dari kamera ke luka dengan referensi fokus. Kedua, kalibrasi pose kamera dari penggabungan data sensor (akselerometer, magnetometer \& giroskop). Ketiga, Jepit \& perbesar untuk mengukur ukuran luka dari referensi yang sebelumnya diketahui \citep{white2014algorithms:16}. Sayangnya, penyimpangan (\emph{drift}) dari ukuran sebenarnya di atas cukup besar karena setiap langkah meningkatkan \emph{drift}. Poon et. al., Yang melanjutkan penelitian mempertahankan fokus yang sama, kecuali jarak luka. Metode batas luka telah diubah menjadi \emph{Grabcut}, maka setiap citra yang diambil diproyeksikan ke bidang 2d untuk menstabilkan sudut. Akhirnya warna tersegmentasi menjadi warna RYB \citep{poon2015algorithms:17}. Salah satu kelemahan dari pendekatan yang diambil, deteksi batas dengan \emph{Grabcut} hanya dapat dijalankan secara semi-otomatis, karena membutuhkan penyesuaian parameter. %fine

Setelah batas luka telah ditentukan dengan benar, dapat digunakan untuk memproduksi \emph{gel bioprinting} untuk penutupan luka. Gholami, \emph{et. al.}, mengevaluasi tujuh algoritma untuk memenuhi tujuan ini \citep{gholami2017segmentation:18}. Tiga algoritma yang dibandingkan adalah dari berbasis tepi, tiga lainnya berdasarkan pertumbuhan wilayah, dan satu lainnya berdasarkan tekstur. Algoritma \emph{Livewire} yang didasarkan pada tepi adalah yang terbaik di antaranya. Berdasarkan kajian teori di atas, \emph{grabcut} digunakan sebagai segmentasi wilayah luka dan warna citra luka dikonversi menjadi \emph{Commission internationale de l'éclairage} (CIE) untuk membuat mekanisme kalibrasi. %fine

Salah satu metode yang banyak digunakan dalam aplikasi pemrosesan citra medis dan biomedis adalah metode kontur aktif (\emph{active contour}) atau yang lebih dikenal dengan sebutan \emph{snake} yang diperkenalkan oleh M. Kass et. al. pada tahun 1988 \citep{kass1988snakes:21}. Sebuah \emph{active contour} (\emph{snake}) adalah kurva yang meminimalkan fungsi energi untuk kondisi tertentu. Fungsi energi ini biasanya terdiri dari dua istilah: energi internal, yang membatasi kelancaran (\emph{smoothness}) dan kekencangan (\emph{tautness}) kontur, dan energi eksternal, yang menarik kontur elastis ke fitur-fitur menarik \citep{acton2007biomedical:19}. Penelitian ini berfokus pada implementasi metode \emph{snake} dalam mendeteksi keliling luka kronis. %fine
% !
%Karena potensi \emph{active contour} yang diusulkan M. Kass et al. tahun 1988, banyak model dari modifikasi \emph{snake} tradisional(dasar), salah satunya adalah \emph{Gradient vector flow} (GVF) yang diusulkan Xu et. al. pada tahun 1998.
% !
Ada kesulitan pada metode \emph{snake} tradisional. Pertama kontur awal yang harus dekat dengan target, dengan kata lain jangkauan tangkap (\emph{capture range}) \emph{snake} terbatas. Masalah kedua ialah \emph{snake} tradisional tidak dapat mengerakkan kontur ke dalam cekungan batas (\emph{concave boundary}) \citep{guo2013review:20} \citep{xu1998snakes:22}. 

%Masalah \emph{capture range} dan \emph{concave boundary} sepenuhnya berhasil diatasi oleh \emph{Gradient vector flow} (GVF) \emph{snake} yang diusulkan oleh Xu dan Prince \citep{xu1998snakes:22}\citep{acton2007biomedical:19}. Dengan menggunakan beberapa contoh dua dimensi dan satu contoh tiga dimensi, menunjukkan bahwa GVF memiliki jangkauan tangkapan besar dan mampu memindahkan ular ke batas cekung\citep{xu1998snakes:22}.
% !
%Meskipun masalah terkait dengan \emph{active contour} sebagian besar telah diatasi oleh Xu et. al. yaitu dengan GVF, akan tetapi metode GVF memiliki kelemahan karena terkadang hasil dari kontur berhenti di tepi palsu (\emph{spurious edges})\citep{abdullah2016robust}. 
%(copy)
%Abdullah, et. al. dalam penelitian tentang segmentasi iris berhasil mengatasi kelemahan dari metode GVF dengan metode yang mereka usulkan, yaitu menambahkan \emph{pressure force} pada GVF. Arah pergerakan \emph{active contour} disesuaikan dengan kelopak mata sehingga menghasilkan hasil yang akurat dan efisien\citep{abdullah2016robust}.
Abdullah, et. al. dalam penelitian tentang segmentasi iris berhasil mengatasi kelemahan dari metode \emph{snake} dengan metode yang mereka usulkan, yaitu menambahkan \emph{pressure force} pada \emph{snake}. Arah pergerakan \emph{snake} disesuaikan dengan kelopak mata sehingga menghasilkan hasil yang akurat dan efisien \citep{abdullah2016robust}.

Penulis tertarik menerapkan metode \emph{snake} pada penelitian ini dengan alasan metode ini adalah metode yang umum digunakan dalam aplikasi pemrosesan citra medis dan biomedis. Selain itu, \emph{snake} cocok untuk mendeteksi objek dengan bentuk bebas \emph{(free-form object)} seperti halnya dengan citra luka kronis. Dikutip dari jurnal \emph{Medical and Biological Image Analysis} tahun 2018, \emph{snake} telah banyak dan baik digunakan dalam berbagai aplikasi seperti segmentasi CT-Scan otak, segmentasi untuk deteksi kanker payudara, deteksi lesi (bintik) pada kulit dan lain-lain \citep{hemalatha2018active}. %fine
% !
Secara ideal, penulis ingin juga menerapkan metode yang dikembangkan oleh \citep{abdullah2016robust}, namun hal tersebut belum dapat dilakukan karena harus memahami lebih terlebih dahulu tentang \emph{snake}, maka dari itu untuk penelitian ini dibatasi menggunakan metode \emph{active contour} saja. Penelitian ini akan berfokus pada pengembangan metode \emph{snake} dalam mendeteksi keliling luka kronis.

\emph{Dataset} citra yang penulis gunakan untuk penelitian ini didapat dari penelitian luka \mbox{Ns. Ratna Aryani, M.Kep}, tahun 2018 \mbox{\citep{ratna2018rancang}} yang tersedia di \emph{repository} \url{https://github.com/mekas/InjuryDetection}. Penelitian ini dilakukan sampai mendapatkan hasil berupa nilai akurasi yang didapat dari selisih area kurva akhir \emph{snake} terhadap luas area \emph{ground truth} (nilai sebenarnya).



\section{Rumusan Masalah}
Berdasarkan Latar belakang yang telah dikemukakan di atas. Fokus permasalahan pada penelitian ini adalah “Bagaimana cara mendeteksi keliling luka kronis menggunakan metode \emph{Active contour} (\emph{Snake}) dan \emph{active contour} yang ditambahkan interpolasi ?”.

\section{Pembatasan Masalah}
\begin{enumerate}
	\item Pendeteksian keliling luka kronis menggunakan \emph{snake} dan \emph{snake} yang ditambahkan interpolasi menggunakan data citra luka yang didapat dari penelitian luka \mbox{Ns. Ratna Aryani, M.Kep}, tahun 2018 \mbox{\citep{ratna2018rancang}}.
	\item Penelitian dilakukan sampai mendapatkan hasil, yaitu nilai akurasi dari selisih area kurva \emph{snake} terhadap area \emph{ground truth}
\end{enumerate}

\section{Tujuan Penelitian}
Tujuan penelitian ini adalah untuk mengetahui hasil dari metode \emph{snake} dan \emph{snake} yang ditambahkan interpolasi dalam mendeteksi keliling luka kronis.

\section{Manfaat Penelitian}
\begin{enumerate}
	\item Bagi peneliti
		
	Penelitian ini merupakan media penerapan ilmu pengetahuan, khususnya dalam pengembangan metode \emph{Active contour} pada pengkajian luka kronis.
		
	\item Instansi terkait 
	 	
	 	Metode yang diajukan diharapkan dapat membuka peluang untuk diajukan ke instansi kesehatan terkait dalam proses pengkajian luka kronis.	
	 	
	 \item Bagi ilmu pengetahuan
	 	\begin{itemize}
	 		\item Mahasiswa
	 			
	 			Diharapkan penelitian ini dapat digunakan sebagai penunjang referensi, khususnya pustaka tentang deteksi keliling menggunakan \emph{Active contour}.
	 			
	 		\item Bagi peneliti selanjutnya
	 			
	 			Diharapkan Penelitian ini dapat digunakan sebgai dasar atau kajian awal bagi peneliti lain yang ingin meneliti permasalahan yang sama.
	 			
	 	\end{itemize}	
\end{enumerate}


% Baris ini digunakan untuk membantu dalam melakukan sitasi
% Karena diapit dengan comment, maka baris ini akan diabaikan
% oleh compiler LaTeX.
\begin{comment}
\bibliography{daftar-pustaka}
\end{comment}
