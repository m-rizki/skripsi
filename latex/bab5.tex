%!TEX root = ./template-skripsi.tex
%-------------------------------------------------------------------------------
%                            	BAB IV
%               		KESIMPULAN DAN SARAN
%-------------------------------------------------------------------------------

\chapter{KESIMPULAN DAN SARAN}

\section{Kesimpulan}
Berdasarkan hasil dari eksperimen, maka dapat ditarik kesimpulan sebagai berikut:
\begin{enumerate}
	\item Kontribusi peneliti terdapat pada penemuan dibutuhkannya tahap interpolasi \emph{preprocessing} sebelum \emph{active contour} dijalankan.
	\item Hasil dari deteksi keliling luka menggunakan \emph{snake} versi \emph{integer} yang kurva akhirnya berhasil menutupi luka berjumlah 12 data (dari 71 data) sedangkan untuk yang versi interpolasi berjumlah 44 data (dari 71 data). Hal ini menunjukkan bahwa data yang berhasil dideteksi menggunakan \emph{snake} interpolasi lebih banyak dibandingkan versi \emph{integer}.
	\item Hasil dari deteksi keliling luka menggunakan \emph{snake} versi \emph{integer} untuk semua kategori menghasilkan 12 data (dari 71 data) yang kurva akhirnya berhasil menutupi luka dengan nilai akurasi rata-rata 77.18\%.
	\item Hasil dari deteksi keliling luka menggunakan \emph{snake} versi \emph{integer} untuk semua kategori menghasilkan 44 data (dari 71 data) yang kurva akhirnya berhasil menutupi luka dengan nilai akurasi rata-rata 86.1\%.
\end{enumerate}

\section{Saran}
Penemuan tahap interpolasi sebelum \emph{snake} dijalankan ini membuka peluang dilakukannya penelitian lanjutan untuk meningkatkan kinerja pada \emph{snake} dengan mengganti atau memodifikasi metode \emph{snake} interpolasi dengan metode \emph{preprocessing} citra lainnya demi meningkatkan hasil akurasi dari \emph{snake}.
% Baris ini digunakan untuk membantu dalam melakukan sitasi
% Karena diapit dengan comment, maka baris ini akan diabaikan
% oleh compiler LaTeX.
\begin{comment}
\bibliography{daftar-pustaka}
\end{comment}